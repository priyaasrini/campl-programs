\usepackage[utf8]{inputenc}

\usepackage{bussproofs}

%%% -- Use biblatex instead of bibtex
%\usepackage[english]{babel}
%\usepackage[style=numeric, sorting=nty]{biblatex}


\usepackage[pdftex,dvipsnames]{xcolor}  % Coloured text etc.
%\usepackage[top=2cm, bottom=2.5cm, right=1.5cm, left=1.5cm]{geometry} 
%\usepackage{breakurl}
\usepackage{amsmath,amsthm,amssymb,amsfonts,graphicx}
\usepackage{xypic}
\usepackage{stmaryrd}
\usepackage{tikz}
\usepackage{proof}
\usepackage[shortlabels]{enumitem}
\usepackage{mathtools}
\usepackage{lscape}
\usepackage{tocloft}
\usepackage{multicol} 
\usepackage{leftidx}
\usepackage{bbm}
\usepackage{rotating}
\usepackage{extarrows}
\usepackage{graphicx}
\usepackage{placeins}
\usepackage{dsfont}
\usepackage{caption}
\providecommand{\event}{ACT 2023}
\usepackage{tikz-cd}
\usepackage[margin=1in, top=1in, bottom=1in]{geometry}
\usepackage{bussproofs}
\usepackage{dutchcal}
\usepackage{newpxtext}
\usepackage[varg,bigdelims]{newpxmath}
\usepackage[backend=biber, backref=true, maxbibnames = 10, style = alphabetic]{biblatex}
\usepackage[bookmarks=true, colorlinks=true, linkcolor=blue!50!black,
citecolor=orange!50!black, urlcolor=orange!50!black, pdfencoding=unicode]{hyperref}
\usepackage[capitalize]{cleveref}
\usepackage{bbm}
\usepackage{floatpag}

%\newtheorem{observation}{dummy}[section]
\newcounter{dummy} 
\numberwithin{dummy}{section}

\newtheorem{lemma}[dummy]{Lemma}  %%share counter with remark
\newtheorem{theorem}[dummy]{Theorem}
\theoremstyle{definition}
\newtheorem{definition}[dummy]{Definition}
\newtheorem{ob}[dummy]{Observation}
\newtheorem{notation}[dummy]{Notation}
\newtheorem{proposition}[dummy]{Proposition} 
\newtheorem{corollary}[dummy]{Corollary} 
\theoremstyle{definition}
\newtheorem{example}[dummy]{Example}
\newtheorem{hypothesis}[dummy]{Hypothesis}
\numberwithin{equation}{section}


\newcommand{\tri
}{\triangleleft
}

\setlength{\cftbeforesecskip}{1pt}
\setlength{\cftbeforesubsecskip}{1pt}
\setlength{\cftbeforesubsubsecskip}{1pt}


\newcommand{\funfact}[1]{\noindent\fbox{\begin{minipage}{\textwidth}
#1
\end{minipage}}}

\newcommand{\s}{{\sf s}}
\renewcommand{\t}{{\sf t}}
\renewcommand{\u}{{\sf{u}}}
\renewcommand{\v}{{\sf{v}}}
\newcommand{\<}{\langle}
\renewcommand{\>}{\rangle}
\newcommand{\X}{\mathbb{X}}
\newcommand{\A}{\mathbb{A}}
\newcommand{\B}{\mathbb{B}}
\newcommand{\C}{\mathbb{C}}
\newcommand{\D}{\mathbb{D}}
\newcommand{\I}{\mathbb{I}}
\newcommand{\J}{\mathbb{J}}
\newcommand{\N}{\mathbb{N}}
\newcommand{\U}{\mathbb{U}}
\newcommand{\V}{\mathbb{V}}
\newcommand{\Y}{\mathbb{Y}}
\newcommand{\Z}{\mathbb{Z}}
\newcommand{\R}{\mathbb{R}}
\newcommand{\dsa}{$\dag$-$*$-autonomous}  
\newcommand{\dldc}{$\dag$-LDC}  
\newcommand{\m}{{\sf m}}
\newcommand{\f}{{\sf f}}

\newcommand{\Set}{{\sf Set}}
\newcommand{\duo}{\mathsf{duo}}
\newcommand{\indep}{{\sf indep}}
\newcommand{\Cocore}{{\sf Cocore}}
\newcommand{\Poly}{{\sf Poly}}
\newcommand{\lCore}{{\sf core_{\ell}}}
\newcommand{\rCore}{{\sf core_r}}
\newcommand{\nD}{{\sf normalDuo}}
\newcommand{\Iso}{{\sf Isomix}}
\newcommand{\coeval}{{\sf coev}}
\newcommand{\eval}{{\sf ev}}

\newcommand{\dashvv}{\dashv \!\!\!\! \dashv}  
\newcommand{\yon}{\mathcal{y}}
\newcommand{\then}{\fatsemi}

\newcommand{\priyaa}[1]{\textcolor{purple}{#1}}

\newcommand{\nat}{\text{nat. }} 
\newcommand{\id}{{\sf id}} 
\newcommand{\CP}{\mathsf{CP}}
\newcommand{\ox}{\otimes}
\newcommand{\pr}{\oplus}
\newcommand{\oa}{\oplus}
\newcommand{\op}{\mathsf{op}}
\newcommand{\rev}{\mathsf{rev}}
\newcommand{\mx}{\mathsf{mx}}
\newcommand{\Chu}{\mathsf{Chu}}
\newcommand{\FRel}{\mathsf{FRel}}
\newcommand{\FMat}{\mathsf{FMat}}
\newcommand{\Rel}{\mathsf{Rel}}
\newcommand{\Mat}{\mathsf{Mat}}
\newcommand{\Core}{\mathsf{Core}}
\newcommand{\FHilb}{\mathsf{FHilb}}
\newcommand{\Unitary}{\mathsf{Unitary}}
\newcommand{\dual}{\mathbin{\text{\reflectbox{$\Vdash$}}}}
\newcommand{\fin}{\mathsf{FinSp}}
\newcommand{\lollipop}{\ensuremath{\!-\!\!\circ}}
\renewcommand{\bar}[1]{\overline{#1}}
\newcommand{\x}{\times}
\newcommand {\poppilol} {\reflectbox{$\multimap$}}


\iffalse
\let\oldsection\section% Store \section
\renewcommand{\section}{% Update \section
  \renewcommand{\theequation}{\thesection.\arabic{equation}}% Update equation number
  \oldsection}% Regular \section
\let\oldsubsection\subsection% Store \subsection
\renewcommand{\subsection}{% Update \subsection
  \renewcommand{\theequation}{\thesubsection.\arabic{equation}}% Update equation number
  \oldsubsection}% Regular \subsection  
\let\oldsubsubsection\subsubsection% Store \subsection
\renewcommand{\subsubsection}{% Update \subsection
  \renewcommand{\theequation}{\thesubsubsection.\arabic{equation}}% Update equation number
  \oldsubsubsection}% Regular \subsection
\fi 

% Jeff Egger's tensor and par
\newlength{\llcfoo}
\def\superimpose#1#2{
  \settowidth{\llcfoo}{#2}
  \makebox[\llcfoo]{\makebox[0pt]{#1}\makebox[0pt]{#2}}}
%% \superimpose assumes that the first argument is narrower (or
%% equal-in-width) to the second.
\def\mathsuperimpose#1#2{\mathchoice{
  \superimpose{\ensuremath{\displaystyle#1}}{\ensuremath{\displaystyle#2}}}{
  \superimpose{\ensuremath{\textstyle#1}}{\ensuremath{\textstyle#2}}}{
  \superimpose{\ensuremath{\scriptstyle#1}}{\ensuremath{\scriptstyle#2}}}{
  \superimpose{\ensuremath{\scriptscriptstyle#1}}{\ensuremath{\scriptscriptstyle#2}}}}

      \def\quasipt{1pt}     	% if 12pt font
%     \def\quasipt{0.75pt}  	% if 11pt font
%     \def\quasipt{0.67pt}  	% if 10pt font
      \def\minipt{0.6pt}	% seems to work reasonably well in all fonts
      \def\tinypt{0.4pt}	% seems to work reasonably well in all fonts
    \def\smalltimes{\raisebox{\quasipt}{$\scriptstyle\times$}}
    \def\tinytimes{\raisebox{\minipt}{$\scriptscriptstyle\times$}}
    \def\teenytimes{\cdot} % or {\scriptscriptstyle\ast}} or similar
    \def\fixnormalcup{\raisebox{-\quasipt}{$\cup$}}
    \def\fixsmallcup{\raisebox{-\minipt}{$\scriptstyle\cup$}}
    \def\fixtinycup{\raisebox{-\tinypt}{$\scriptscriptstyle\cup$}}
  \def\fixtimes{
    \mathchoice{\smalltimes}{\smalltimes}{\tinytimes}{\teenytimes}}
  \def\fixcup{
    \mathchoice{\fixnormalcup}{\fixnormalcup}{\fixsmallcup}{\fixtinycup}}
\def\ocap{\mathrel{\mathsuperimpose{\fixtimes}{\cap}}}
\def\ocup{\mathrel{\mathsuperimpose{\fixtimes}{\fixcup}}}
\def\bigocap{\mathop{\mathsuperimpose{\times}{\bigcap}}\limits}
\def\bigocup{\mathop{\mathsuperimpose{\times}{\bigcup}}\limits}
\def\bip{\mathop{\mathsuperimpose{\times}{+}}\limits}


%%%%%%%%%%%%%%%%%%%%%%%%%%%%%%%%%%%%%%%%%%%%%%%%%%%%%%%%%%%%%%%%%%%%%%%%%
% M. Barr uses the following:  "It gives a \to that can be used as
% $A\to B$ or $A\to^f B$ or $A\to^{f\o g\o h}B$ or even $A\to^f_gB$.  The
% arrow will grow to fit the label(s).  There are similar definitions for
% \two and \tofro, for which you really might want labels both above and
% below.  Actually, by reading your definition of \kto, I was able to
% simplify this.  But it is still nice to have the optional arguments.
% There is only caveat: although you can have one or the other or both
% labels, if you have both the upper must precede the lower.  These defs
% must either be placed in a style file xor surrounded by \makeatletter
% and \makeatother (but NOT both)."  (Modifications by rags)
% The definitions below look more elaborate than they need to be.
% The reason is that an empty asscript will still cause extra vertical
% spacing and the only way to avoid ugly extra space seems to be using
% some such method as this.

\makeatletter
\newenvironment{myproof}[1][\proofname]{\par
    \pushQED{\qed}%
    \normalfont \topsep6\p@\@plus6\p@\relax
    \trivlist
    \item[\hskip\labelsep
        \itshape
        #1\@addpunct{.} ]\mbox{}\par\nobreak}
    {\popQED\endtrivlist\@endpefalse}
\makeatother

\makeatletter

% In-text size:

\newdimen\w@dth

\def\setw@dth#1#2{\setbox\z@\hbox{\scriptsize $#1$}\w@dth=\wd\z@
\setbox\@ne\hbox{\scriptsize $#2$}\ifnum\w@dth<\wd\@ne \w@dth=\wd\@ne \fi
\advance\w@dth by 1.2em}

\def\t@^#1_#2{\allowbreak\def\n@one{#1}\def\n@two{#2}\mathrel
{\setw@dth{#1}{#2}
\mathop{\hbox to \w@dth{\rightarrowfill}}\limits
\ifx\n@one\empty\else ^{\box\z@}\fi
\ifx\n@two\empty\else _{\box\@ne}\fi}}
\def\t@@^#1{\@ifnextchar_ {\t@^{#1}}{\t@^{#1}_{}}}


\def\t@left^#1_#2{\def\n@one{#1}\def\n@two{#2}\mathrel{\setw@dth{#1}{#2}
\mathop{\hbox to \w@dth{\leftarrowfill}}\limits
\ifx\n@one\empty\else ^{\box\z@}\fi
\ifx\n@two\empty\else _{\box\@ne}\fi}}
\def\t@@left^#1{\@ifnextchar_ {\t@left^{#1}}{\t@left^{#1}_{}}}


\def\two@^#1_#2{\def\n@one{#1}\def\n@two{#2}\mathrel{\setw@dth{#1}{#2}
\mathop{\vcenter{\hbox to \w@dth{\rightarrowfill}\kern-1.7ex
                 \hbox to \w@dth{\rightarrowfill}}%
       }\limits
\ifx\n@one\empty\else ^{\box\z@}\fi
\ifx\n@two\empty\else _{\box\@ne}\fi}}
\def\tw@@^#1{\@ifnextchar_ {\two@^{#1}}{\two@^{#1}_{}}}


\def\tofr@^#1_#2{\def\n@one{#1}\def\n@two{#2}\mathrel{\setw@dth{#1}{#2}
\mathop{\vcenter{\hbox to \w@dth{\rightarrowfill}\kern-1.7ex
                 \hbox to \w@dth{\leftarrowfill}}%
       }\limits
\ifx\n@one\empty\else ^{\box\z@}\fi
\ifx\n@two\empty\else _{\box\@ne}\fi}}
\def\t@fr@^#1{\@ifnextchar_ {\tofr@^{#1}}{\tofr@^{#1}_{}}}

% Displaysize:

\newdimen\W@dth
\def\setW@dth#1#2{\setbox\z@\hbox{$#1$}\W@dth=\wd\z@
\setbox\@ne\hbox{$#2$}\ifnum\W@dth<\wd\@ne \W@dth=\wd\@ne \fi
\advance\W@dth by 1.2em}

\def\T@^#1_#2{\allowbreak\def\N@one{#1}\def\N@two{#2}\mathrel
{\setW@dth{#1}{#2}
\mathop{\hbox to \W@dth{\rightarrowfill}}\limits
\ifx\N@one\empty\else ^{\box\z@}\fi
\ifx\N@two\empty\else _{\box\@ne}\fi}}
\def\T@@^#1{\@ifnextchar_ {\T@^{#1}}{\T@^{#1}_{}}}


\def\T@left^#1_#2{\def\N@one{#1}\def\N@two{#2}\mathrel{\setW@dth{#1}{#2}
\mathop{\hbox to \W@dth{\leftarrowfill}}\limits
\ifx\N@one\empty\else ^{\box\z@}\fi
\ifx\N@two\empty\else _{\box\@ne}\fi}}
\def\T@@left^#1{\@ifnextchar_ {\T@left^{#1}}{\T@left^{#1}_{}}}


\def\Tofr@^#1_#2{\def\N@one{#1}\def\N@two{#2}\mathrel{\setW@dth{#1}{#2}
\mathop{\vcenter{\hbox to \W@dth{\rightarrowfill}\kern-1.7ex
                 \hbox to \W@dth{\leftarrowfill}}%
       }\limits
\ifx\N@one\empty\else ^{\box\z@}\fi
\ifx\N@two\empty\else _{\box\@ne}\fi}}
\def\T@fr@^#1{\@ifnextchar_ {\Tofr@^{#1}}{\Tofr@^{#1}_{}}}


\def\Two@^#1_#2{\def\N@one{#1}\def\N@two{#2}\mathrel{\setW@dth{#1}{#2}
\mathop{\vcenter{\hbox to \W@dth{\rightarrowfill}\kern-1.7ex
                 \hbox to \W@dth{\rightarrowfill}}%
       }\limits
\ifx\N@one\empty\else ^{\box\z@}\fi
\ifx\N@two\empty\else _{\box\@ne}\fi}}
\def\Tw@@^#1{\@ifnextchar_ {\Two@^{#1}}{\Two@^{#1}_{}}}


\def\to{\@ifnextchar^ {\t@@}{\t@@^{}}}
\def\from{\@ifnextchar^ {\t@@left}{\t@@left^{}}}
\def\tofro{\@ifnextchar^ {\t@fr@}{\t@fr@^{}}}
\def\To{\@ifnextchar^ {\T@@}{\T@@^{}}}
\def\From{\@ifnextchar^ {\T@@left}{\T@@left^{}}}
\def\Two{\@ifnextchar^ {\Tw@@}{\Tw@@^{}}}
\def\Tofro{\@ifnextchar^ {\T@fr@}{\T@fr@^{}}}

\makeatother
\newcommand{\vcenteredinclude}[2]{\begingroup
\setbox0=\hbox{\includegraphics[#1]{#2}}%
\parbox{\wd0}{\box0}\endgroup}
% for pullback corner
\newcommand{\pullbackcorner}[1][ul]{\save*!/#1+1.2pc/#1:(1,-1)@^{|-}\restore}
\newcommand{\pushoutcorner}[1][dr]{\save*!/#1-1.2pc/#1:(-1,1)@^{|-}\restore}

%%%%%%%%%%%%%%%%%% TikZ %%%%%%%%%%%%%%%%%%%%%%
\tikzstyle{strings}=[baseline={([yshift=-.5ex]current bounding box.center)}]

%Global tikz scaling

\tikzset{every picture/.append style={scale=.5}, transform shape, strings}

\tikzset{%
symbol/.style={%
draw=none,
every to/.append style={%
edge node={node [sloped, allow upside down, auto=false]{$#1$}}}
}
}

\usetikzlibrary{shapes.geometric}
\usetikzlibrary{patterns}
\usetikzlibrary{fit}
\usetikzlibrary{positioning}
\usetikzlibrary{calc}
\usetikzlibrary{arrows}
\usetikzlibrary{decorations.markings}
\usetikzlibrary{decorations.pathreplacing}
\usetikzlibrary{shapes}

%% -------------------------------------- Declare the layers
\pgfdeclarelayer{nodelayer}
\pgfdeclarelayer{edgelayer}
\pgfsetlayers{edgelayer,nodelayer,main}


%% -------------------------------------- Declare the styles
\tikzset{simple/.style={}}
\tikzset{nothing/.style={outer sep=-3.4pt}}

\tikzset{map/.style={draw,fill=white, thick, rectangle}}
\tikzset{mapblack/.style={draw,fill=black, rectangle}}

% Edge styles
\tikzstyle{filled}=[-, fill=black]

\tikzset{dot/.style={thick, fill=black, circle, scale=1, inner sep = .05cm}}

\tikzset{oa/.style={draw, scale=0.9,minimum height=.1cm,circle,append after command={
[shorten >=\pgflinewidth, shorten <=\pgflinewidth,]
(\tikzlastnode.north) edge (\tikzlastnode.south)
(\tikzlastnode.east) edge (\tikzlastnode.west)
} } }

\tikzset{ox/.style={draw, scale=0.9,minimum height=.1cm,circle,append after command={
[shorten >=\pgflinewidth, shorten <=\pgflinewidth,]
(\tikzlastnode.north west) edge (\tikzlastnode.south east)
(\tikzlastnode.north east) edge (\tikzlastnode.south west) } } }


\tikzset{coprod/.style={draw, thick, scale=0.75, circle } }

\tikzset{prod/.style={draw, fill=black, scale=0.75, circle } }


\tikzset{circ/.style={
shape=circle, inner sep=1pt, draw}}

% Styles added by Priyaa
\tikzstyle{none}=[inner sep=-1pt]
\tikzstyle{circle}=[shape=circle,draw]

\tikzstyle{onehalfcircle}=[shape=circle, scale=1.5, draw]
\tikzstyle{twocircle}=[shape=circle, scale=2, draw]
\tikzstyle{black}=[shape=circle, fill=black, draw]

\tikzstyle{head}=[fill=white, draw=black, shape=circle, scale=2, thick]
\tikzstyle{grayhead}=[dashed, fill=white, draw=gray, shape=circle, scale=2]
\tikzstyle{process}=[fill=white, draw=black, shape=circle, scale=2, thick]


\newcommand*{\StrikeThruDistance}{0.15cm}%
\newcommand*{\StrikeThru}{\StrikeThruDistance,\StrikeThruDistance}%

\tikzset{wires/.style={}}

\tikzset{box/.style={inner sep=0pt, thick, draw=black, text height=1.5ex, text depth=.25ex, 
text centered, minimum height=3em, anchor=center}}

%%%%%%%%%%%%%%%%%%%%%%%%%%%%%%%%%%%%%%%%%%%%%%%%%%%%%%%%%%%%%%%%%%%%%%%%

\tikzcdset{every label/.append style = {font = \Large}}

%%%%%%%%%%%%%%%%%%%%%%%%%%%%%%%%%%%%%%%%%%%%%%%%%%%%%%%%%%%%%%%%%%%%%%%%

\newcommand{\linmonw} {\xymatrixcolsep{4mm} \xymatrix{ \ar@{-||}[r]^{\circ} & }}
%\newcommand{\linmonwr} {\xymatrixcolsep{4mm} \xymatrix{ \ar@{-||}[r]^{\triangleright} & }}
\newcommand{\linmonwl} {\xymatrixcolsep{4mm} \xymatrix{ \ar@{-||}[r]^{\otimes\;\tri} & }}
\newcommand{\linmonwr} {\xymatrixcolsep{4mm} \xymatrix{ \ar@{-||}[r]^{\tri\;\otimes} & }}
\newcommand{\linmonwrdavid} {\xymatrixcolsep{4mm} \xymatrix{ \ar@{-||}[r]^{\otimes\;\tri} & }}
\newcommand{\linmonwldavid} {\xymatrixcolsep{4mm} \xymatrix{ \ar@{-||}[r]^{\tri\;\otimes} & }}
\newcommand{\linmondavid} {\xymatrixcolsep{4mm} \xymatrix{ \ar@{-||}[r] & }}

\newcommand{\lincomonb} {\xymatrixcolsep{4mm} \xymatrix{ \ar@{-||}[r]_{\bullet} & }}
\newcommand{\lincomonw} {\xymatrixcolsep{4mm} \xymatrix{ \ar@{-||}[r]_{\circ} & }}
\newcommand{\lincomonwr} {\xymatrixcolsep{4mm} \xymatrix{ \ar@{-||}[r]_{\tri\;\otimes} & }}
\newcommand{\lincomonwl} {\xymatrixcolsep{4mm} \xymatrix{ \ar@{-||}[r]_{\otimes\;\tri} & }}

\newcommand{\linbialgw} {\xymatrixcolsep{4mm} \xymatrix{ \ar@{-||}[r]^{\circ}_{\circ} & }}
\newcommand{\linbialgwl} {\xymatrixcolsep{4mm} \xymatrix{ \ar@{-||}[r]^{\otimes\;\tri}_{\otimes\;\tri} & }}
\newcommand{\linbialgwr} {\xymatrixcolsep{4mm} \xymatrix{ \ar@{-||}[r]^{\tri\;\otimes}_{\tri\;\otimes} & }}
\newcommand{\linbialgwb} {\xymatrixcolsep{4mm} \xymatrix{ \ar@{-||}[r]^{\circ}_{\bullet} & }}


\newcommand{\monoid}[1]{(#1, \mulmap{1.5}{white}: #1 \ox #1 \to #1, \unitmap{1.5}{white}: \yon \to #1)}
\newcommand{\comonoid}[1]{(#1, \comulmap{1.5}{white}: #1 \to #1 \ox #1, \counitmap{1.5}{white}: #1 \to \yon)}
\newcommand{\comonoidb}[1]{(#1, \comulmap{1.5}{black}: #1 \to #1 \ox #1, \counitmap{1.5}{black}: #1 \to \yon)}
\newcommand{\Frob}[1]{(#1, \mulmap{1.5}{white}, \unitmap{1.5}{white}, \comulmap{1.5}{white}, \counitmap{1.5}{white})}
\newcommand{\bFrob}[1]{(#1, \mulmap{1.5}{black}, \unitmap{1.5}{black}, \comulmap{1.5}{black}, \counitmap{1.5}{black})}
\newcommand{\bialg}[1]{(#1, \mulmap{1.5}{white}, \unitmap{1.5}{white}, \comulmap{1.5}{black}, \counitmap{1.5}{black})}
\newcommand{\bialgb}[1]{(#1, \mulmap{1.5}{black}, \unitmap{1.5}{black}, \comulmap{1.5}{white}, \counitmap{1.5}{white})}
\newcommand{\trimonoid}[1]{(#1, \trianglemult{0.65}: #1 \tri #1 \to #1, \triangleunit{0.65}: \yon \to #1)}
\newcommand{\tricomonoid}[1]{(#1, \trianglecomult{0.65}: #1 \to #1 \tri #1, \trianglecounit{0.65}: #1 \to \yon)}

\newcommand{\blackman}{
\begin{tikzpicture}[scale=1.5]
	\begin{pgfonlayer}{nodelayer}
		\node [style=head] (0) at (-3, 5) {};
		\node [style=none] (1) at (-3, 4) {};
		\node [style=none] (2) at (-3.25, 3.75) {};
		\node [style=none] (3) at (-2.75, 3.75) {};
		\node [style=none] (4) at (-3.25, 4.45) {};
		\node [style=none] (5) at (-2.75, 4.45) {};
	\end{pgfonlayer}
	\begin{pgfonlayer}{edgelayer}
		\draw[thick] (0) to (1.center);
		\draw[thick] (1.center) to (2.center);
		\draw[thick] (1.center) to (3.center);
		\draw[thick] (4.center) to (5.center);
	\end{pgfonlayer}
\end{tikzpicture} }

\newcommand{\grayman}{
\begin{tikzpicture}[scale=1.5]
	\begin{pgfonlayer}{nodelayer}
		\node [style=grayhead,thick] (0) at (-3, 5) {};
		\node [style=none] (1) at (-3, 4) {};
		\node [style=none] (2) at (-3.25, 3.75) {};
		\node [style=none] (3) at (-2.75, 3.75) {};
		\node [style=none] (4) at (-3.25, 4.45) {};
		\node [style=none] (5) at (-2.75, 4.45) {};
	\end{pgfonlayer}
	\begin{pgfonlayer}{edgelayer}
		\draw[thick, color=gray, dashed] (0) to (1.center);
		\draw[thick, color=gray, dashed] (1.center) to (2.center);
		\draw[thick, color=gray, dashed] (1.center) to (3.center);
		\draw[thick, color=gray, dashed] (4.center) to (5.center);
	\end{pgfonlayer}
\end{tikzpicture}  }



\newcommand{\linbialgwtik} {\begin{tikzpicture}
	\begin{pgfonlayer}{nodelayer}
		\node [style=none] (0) at (-2.8, 1.17) {};
		\node [style=none] (1) at (-1.85, 1.17) {};
		\node [style=none] (2) at (-2, 1.35) {};
		\node [style=none] (3) at (-2, 1) {};
		\node [style=none] (4) at (-1.85, 1) {};
		\node [style=none] (5) at (-1.85, 1.35) {};
		\node [style=none] (6) at (-2.2, 1) {};
		\node [style=none] (7) at (-2.5, 1) {};
		\node [style=none] (8) at (-2.35, 0.82) {};
		\node [style=none] (9) at (-1.6, 1.17) {};
		\node [style=none] (10) at (-3.05, 1.17) {};
		\node [style=circle, scale=0.6] (11) at (-2.35, 1.45) {};
		%\node [style=none] (12) at (-2.25, 0.5) {}; %extra node for spacing
	\end{pgfonlayer}
	\begin{pgfonlayer}{edgelayer}
		\draw (2.center) to (3.center);
		\draw (5.center) to (4.center);
		\draw (0.center) to (1.center);
		\draw (6.center) -- (7.center) -- (8.center) -- (6.center);
	\end{pgfonlayer}
\end{tikzpicture}}


\newcommand{\linmonwtik} {\begin{tikzpicture}
	\begin{pgfonlayer}{nodelayer}
		\node [style=none] (0) at (-2.7, 1.17) {};
		\node [style=none] (1) at (-1.85, 1.17) {};
		\node [style=none] (2) at (-2, 1.35) {};
		\node [style=none] (3) at (-2, 1) {};
		\node [style=none] (4) at (-1.85, 1) {};
		\node [style=none] (5) at (-1.85, 1.35) {};
		\node [style=circle, scale=0.6] (6) at (-2.35, 1.45) {};
		\node [style=none] (7) at (-1.6, 1.17) {};
		\node [style=none] (8) at (-2.95, 1.17) {};
	\end{pgfonlayer}
	\begin{pgfonlayer}{edgelayer}
		\draw (2.center) to (3.center);
		\draw (5.center) to (4.center);
		\draw (0.center) to (1.center);
	\end{pgfonlayer}
\end{tikzpicture}}

\newcommand{\tricomul}[1]{\trianglecomult{#1}}
\newcommand{\trimul}[1]{\trianglemul{#1}}
\newcommand{\tricounit}[1]{\trianglecounit{#1}}
\newcommand{\triunit}[1]{\triangleunit{#1}}

% for multiplication and comultiplication maps
% arguments - scale and color of dot
\newcommand{\mulmap}[2]{
	\begin{tikzpicture}[scale={#1}]
		\begin{pgfonlayer}{nodelayer}
			\node [style=circle, scale=0.4, fill={#2}] (5) at (0.32, 0.25) {};
			\node [style=none] (6) at (0.07, 0.5) {};
			\node [style=none] (7) at (0.57, 0.5) {};
			\node [style=none] (8) at (0.32, 0) {};
			\node [style=none] (9) at (0.64, 0.5) {};
		\end{pgfonlayer}
		\begin{pgfonlayer}{edgelayer}
			\draw [style=none] (8.center) to (5);
			\draw [style=none, bend left, looseness=1.25] (5) to (6.center);
			\draw [style=none, bend right, looseness=1.25] (5) to (7.center);
		\end{pgfonlayer}
	\end{tikzpicture}	
}

% need to specify scale and color of dot
\newcommand{\unitmap}[2]{
\begin{tikzpicture}[scale=#1]
	\begin{pgfonlayer}{nodelayer}
		\node [style=circle, scale=0.4, fill=#2] (0) at (0, 0) {};
		\node [style=none] (1) at (0, -0.4) {};
		\node [style=none] (4) at (0.13, 0) {};
	\end{pgfonlayer}
	\begin{pgfonlayer}{edgelayer}
		\draw [style=none] (0) to (1.center);
	\end{pgfonlayer}
\end{tikzpicture} }

\newcommand{\trianglecomult}[1]{
\begin{tikzpicture}[scale=#1]
	\begin{pgfonlayer}{nodelayer}
		\node [style=none] (0) at (-0.25, 3.75) {};
		\node [style=none] (1) at (-0.5, 3.5) {};
		\node [style=none] (2) at (0, 3.5) {};
		\node [style=none] (3) at (-0.25, 4.25) {};
		\node [style=none] (4) at (0.25, 3) {};
		\node [style=none] (5) at (-0.75, 3) {};
	\end{pgfonlayer}
	\begin{pgfonlayer}{edgelayer}
		\draw [bend left, looseness=1.00] (2.center) to (4.center);
		\draw (0.center) to (1.center);
		\draw (0.center) to (2.center);
		\draw (2.center) to (1.center);
		\draw [in=90, out=-165, looseness=0.75] (1.center) to (5.center);
		\draw (0.center) to (3.center);
	\end{pgfonlayer}
\end{tikzpicture}
}


\newcommand{\trianglecounit}[1]{
\begin{tikzpicture}[scale=#1]
	\begin{pgfonlayer}{nodelayer}
		\node [style=none] (0) at (-0.25, 3.5) {};
		\node [style=none] (1) at (-0.5, 3.25) {};
		\node [style=none] (2) at (0, 3.25) {};
		\node [style=none] (3) at (-0.25, 4.25) {};
		\node [style=none] (4) at (-0.25, 2.8) {};
	\end{pgfonlayer}
	\begin{pgfonlayer}{edgelayer}
		\draw (0.center) to (1.center);
		\draw (0.center) to (2.center);
		\draw (2.center) to (1.center);
		\draw (0.center) to (3.center);
	\end{pgfonlayer}
\end{tikzpicture}~\!\!}

\newcommand{\trianglemult}[1]{
\begin{tikzpicture}[scale=#1]
	\begin{pgfonlayer}{nodelayer}
		\node [style=none] (0) at (-0.25, 3.5) {};
		\node [style=none] (1) at (-0.5, 3.75) {};
		\node [style=none] (2) at (0, 3.75) {};
		\node [style=none] (3) at (-0.25, 3) {};
		\node [style=none] (4) at (0.25, 4.25) {};
		\node [style=none] (5) at (-0.75, 4.25) {};
	\end{pgfonlayer}
	\begin{pgfonlayer}{edgelayer}
		\draw [bend right, looseness=1.00] (2.center) to (4.center);
		\draw (0.center) to (1.center);
		\draw (0.center) to (2.center);
		\draw (2.center) to (1.center);
		\draw [in=-90, out=165, looseness=0.75] (1.center) to (5.center);
		\draw (0.center) to (3.center);
	\end{pgfonlayer}
\end{tikzpicture} }

\newcommand{\triangleunit}[1]{
\begin{tikzpicture}[scale=#1]
	\begin{pgfonlayer}{nodelayer}
		\node [style=none] (0) at (-0.25, 4) {};
		\node [style=none] (1) at (-0.5, 4.25) {};
		\node [style=none] (2) at (0, 4.25) {};
		\node [style=none] (3) at (-0.25, 3.25) {};
		\node [style=none] (4) at (-0.25, 3) {};
	\end{pgfonlayer}
	\begin{pgfonlayer}{edgelayer}
		\draw (0.center) to (1.center);
		\draw (0.center) to (2.center);
		\draw (2.center) to (1.center);
		\draw (0.center) to (3.center);
	\end{pgfonlayer}
\end{tikzpicture}~\!\!}

\newcommand{\comulmap}[2]{
	\begin{tikzpicture}[scale={#1}]
		\begin{pgfonlayer}{nodelayer}
			\node [style=circle, scale=0.4, fill={#2}] (5) at (0.32, 0.25) {};
			\node [style=none] (6) at (0.07, 0) {};
			\node [style=none] (7) at (0.57, 0) {};
			\node [style=none] (8) at (0.32, 0.5) {};
			\node [style=none] (9) at (0.64, 0) {};
		\end{pgfonlayer}
		\begin{pgfonlayer}{edgelayer}
			\draw [style=none] (8.center) to (5);
			\draw [style=none, bend right, looseness=1.25] (5) to (6.center);
			\draw [style=none, bend left, looseness=1.25] (5) to (7.center);
		\end{pgfonlayer}
	\end{tikzpicture}
}

% need to specify scale and color of dot
\newcommand{\counitmap}[2]{
\begin{tikzpicture}[scale=#1, rotate=180]
	\begin{pgfonlayer}{nodelayer}
		\node [style=circle, scale=0.4, fill=#2] (0) at (0, 0) {};
		\node [style=none] (1) at (0, -0.4) {};
		\node [style=none] (4) at (0.13, 0) {};
	\end{pgfonlayer}
	\begin{pgfonlayer}{edgelayer}
		\draw [style=none] (0) to (1.center);
	\end{pgfonlayer}
\end{tikzpicture}
}

\newcommand{\pnote}[1]{{\quad \color{blue}$\lozenge$\;Priyaa says:}~#1\;{\color{blue}$\lozenge$}\quad}
\newcommand{\rnote}[1]{{\quad \color{red}$\lozenge$\;Robin says:}~#1\;{\color{red}$\lozenge$}\quad}

\newcommand{\gray}[1]{\textcolor{gray}{#1}}

\newcommand{\biglens}[2]{
     \begin{bmatrix}{\vphantom{f_f^f}#2} \\ {\vphantom{f_f^f}#1} \end{bmatrix}
}
\newcommand{\littlelens}[2]{
     \begin{bsmallmatrix}{\vphantom{f}#2} \\ {\vphantom{f}#1} \end{bsmallmatrix}
}
\newcommand{\coclose}[2]{
  \relax\if@display
     \biglens{#2}{#1}
  \else
     \littlelens{#2}{#1}
  \fi
}


\newcommand{\qqand}{\qquad\textnormal{and}\qquad}

\newcommand{\mygray}[1]{{\color{gray}#1}}